\documentclass[12pt,letterpaper,notitlepage]{article}
\renewcommand{\familydefault}{\sfdefault}  % sans-serif font (close to Arial)


\title{OPORD 19-001 CAJUN RAIN}
\author{CPT (P) Obvious}
\date{\today}


\usepackage[shortlabels]{enumitem} % lettered numbering
\usepackage{lastpage}
\usepackage[USenglish]{babel}
\usepackage[showseconds=false,useregional=text,calc]{datetime2}
\DTMnewdatestyle
{mildate}% label
{% definitions
	\renewcommand*{\DTMdisplaydate}[4]{%
		\DTMtwodigits{##3} \Milshortmonthname{##2} ##1} % <========= 1.year 2.month 3.day 4.dow
	\renewcommand*{\DTMDisplaydate}{\DTMdisplaydate}
}
\DTMnewdatestyle
{mildatetime}% label
{% definitions
	\renewcommand*{\DTMdisplay}[9]{%
	% <=== 1.year 2.mon 3.day 4.dow 5.hh 6.mm 7.ss 8.TZh 9.TZm
		\DTMtwodigits{##3}##5##6\Miltimezone{##8}\Milshortmonthname{##2}##1} 
	\renewcommand*{\DTMDisplay}{\DTMdisplay}
}
\newcommand{\Milshortmonthname}[1]{%
	\ifcase#1% 0
	  \or JAN%
	  \or FEB%
	  \or MAR%
	  \or APR%
	  \or MAY%
	  \or JUN%
	  \or JUL%
	  \or AUG%
	  \or SEP%
	  \or OCT%
	  \or NOV%
	  \or DEC%
	\fi
}

\newcommand{\addOffset}[1]{%
	\number\numexpr#1+12\relax%
	}
\newcommand{\Miltimezone}[1]{%
	\ifcase\number\addOffset{#1}%
	      Y% UTC-12
	  \or X% UTC-11
	  \or W% UTC-10
	  \or V% UTC-9
	  \or U% UTC-8
	  \or T% UTC-7
	  \or S% UTC-6
	  \or R% UTC-5
	  \or Q% UTC-4
	  \or P% UTC-3
	  \or O% UTC-2
	  \or N% UTC-1
	  \or Z% UTC
	  \or A% UTC+1
	  \or B% UTC+2
	  \or C% UTC+3
	  \or D% UTC+4
	  \or E% UTC+5
	  \or F% UTC+6
	  \or G% UTC+7
	  \or H% UTC+8
	  \or I% UTC+9
	  \or K% UTC+10
	  \or L% UTC+11
	  \or M% UTC+12
	\fi
}


% create shortcuts for classification, units, etc.
\newcommand{\PM}[1][U]{(#1) }  % call with your portion mark, else defaults to (U)
\newcommand{\overallclass}{UNCLASSIFIED: EXERCISE EXERCISE EXERCISE}
\newcommand{\uptwo}{120$^{th}$ CA BDE}
\newcommand{\upone}{560$^{th}$ CA BN}
\newcommand{\unit}{A/560 CA BN (A)}
\newcommand{\tzoffset}{-5}
\newcommand{\tzname}{ROMEO (UTC\tzoffset)}

% custom header
\usepackage{fancyhdr}
\setlength{\headheight}{15.2pt}
\pagestyle{fancy}
\lhead{}
\chead{\overallclass{}}
\rhead{}
\lfoot{}
\cfoot{\thepage{} of \pageref{LastPage} }
\renewcommand{\headrulewidth}{0pt}  % no horizontal line

% ##### TODO ######
% modify enumerated numbering scheme to go 1. | a. | (1) | (a)? 
% re-work bibliography so that it's generated at the top (references), but
%% that the references section tracks what references are MADE in the doc

\begin{document}
\DTMsetdatestyle{mildatetime}

\begin{flushright}
COPY \rule{1cm}{0.4pt} OF \rule{1cm}{0.4pt} COPIES \linebreak
HQ, 3/10 MTN \linebreak
Fort Polk, LA 71459 \linebreak
%\DTMnow
\today
\end{flushright}

\textbf{\PM TIME ZONE USED THROUGHOUT ORDER: \tzname}

\textbf{\PM REFERENCES:}
\begin{enumerate}[(a)]
\item \PM 1/25 SBCT WARNO 01 to OPORD 18-019 (ARCTIC ANVIL)
\item \PM USARAK OPORD 17-038 (ARCTIC ANVIL 19-01)
\item \PM 1/25 SBCT OPORD 18-019 (ARCTIC ANVIL)
\item \PM FRAGO 1 to 1/25 SBCT OPORD 18-019 (ARCTIC ANVIL)
\item \PM FRAGO 2 to 1/25 SBCT OPORD 18-019 (ARCTIC ANVIL)
\end{enumerate}

\textbf{\PM TASK ORGANIZATION:}

\PM \unit{} is OPCON to 3/10$^{th}$ effective 
\DTMdisplay{2019}{10}{01}{-1}{00}{01}{00}{\tzoffset}{00}

\PM And the timezone \tzname is \Miltimezone{\tzoffset}

\PM[U//FOUO] And this is a non-default PM with brackets

\PM{And this is a default PM wrapping the portion}

\section{\PM Situation}
\subsection{\PM Enemy Forces}
\subsection{\PM Friendly Forces}
\subsubsection{\PM Higher Headquarters Two Levels Up. \uptwo{}}
\subsubsection{\PM Higher Headquarters One Level Up. \upone{}}

\section{\PM Mission}
\PM \unit{}
is mobilized and prepared to deploy 
to King George County
%NET \DTMdisplaydate{2019}{10}{15}{-1}
NET \DTMdisplay{2019}{10}{15}{-1}{00}{01}{00}{+0}{00}
IOT conduct stability operations and neutralize threats to King George civil
	administration and soveriegnty under ROA. (Yes you \emph{will} move at 
	Zulu time)

\section{\PM Execution}
\subsection{\PM Commander's Intent}
\subsection{\PM Endstate}
\subsection{\PM Tasks to Subordinate Units}
\subsection{\PM Coordinating Instructions}
\subsection{\PM Timeline}

\section{\PM Sustainment}

\section{\PM Command and Signal}
\subsection{\PM Command}
\subsection{\PM Control}
\subsection{\PM Signal}


\end{document}
